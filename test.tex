% Options for packages loaded elsewhere
\PassOptionsToPackage{unicode}{hyperref}
\PassOptionsToPackage{hyphens}{url}
%
\documentclass[
]{article}
\usepackage{amsmath,amssymb}
\usepackage{iftex}
\ifPDFTeX
  \usepackage[T1]{fontenc}
  \usepackage[utf8]{inputenc}
  \usepackage{textcomp} % provide euro and other symbols
\else % if luatex or xetex
  \usepackage{unicode-math} % this also loads fontspec
  \defaultfontfeatures{Scale=MatchLowercase}
  \defaultfontfeatures[\rmfamily]{Ligatures=TeX,Scale=1}
\fi
\usepackage{lmodern}
\ifPDFTeX\else
  % xetex/luatex font selection
\fi
% Use upquote if available, for straight quotes in verbatim environments
\IfFileExists{upquote.sty}{\usepackage{upquote}}{}
\IfFileExists{microtype.sty}{% use microtype if available
  \usepackage[]{microtype}
  \UseMicrotypeSet[protrusion]{basicmath} % disable protrusion for tt fonts
}{}
\makeatletter
\@ifundefined{KOMAClassName}{% if non-KOMA class
  \IfFileExists{parskip.sty}{%
    \usepackage{parskip}
  }{% else
    \setlength{\parindent}{0pt}
    \setlength{\parskip}{6pt plus 2pt minus 1pt}}
}{% if KOMA class
  \KOMAoptions{parskip=half}}
\makeatother
\usepackage{xcolor}
\usepackage[margin=1in]{geometry}
\usepackage{color}
\usepackage{fancyvrb}
\newcommand{\VerbBar}{|}
\newcommand{\VERB}{\Verb[commandchars=\\\{\}]}
\DefineVerbatimEnvironment{Highlighting}{Verbatim}{commandchars=\\\{\}}
% Add ',fontsize=\small' for more characters per line
\usepackage{framed}
\definecolor{shadecolor}{RGB}{248,248,248}
\newenvironment{Shaded}{\begin{snugshade}}{\end{snugshade}}
\newcommand{\AlertTok}[1]{\textcolor[rgb]{0.94,0.16,0.16}{#1}}
\newcommand{\AnnotationTok}[1]{\textcolor[rgb]{0.56,0.35,0.01}{\textbf{\textit{#1}}}}
\newcommand{\AttributeTok}[1]{\textcolor[rgb]{0.13,0.29,0.53}{#1}}
\newcommand{\BaseNTok}[1]{\textcolor[rgb]{0.00,0.00,0.81}{#1}}
\newcommand{\BuiltInTok}[1]{#1}
\newcommand{\CharTok}[1]{\textcolor[rgb]{0.31,0.60,0.02}{#1}}
\newcommand{\CommentTok}[1]{\textcolor[rgb]{0.56,0.35,0.01}{\textit{#1}}}
\newcommand{\CommentVarTok}[1]{\textcolor[rgb]{0.56,0.35,0.01}{\textbf{\textit{#1}}}}
\newcommand{\ConstantTok}[1]{\textcolor[rgb]{0.56,0.35,0.01}{#1}}
\newcommand{\ControlFlowTok}[1]{\textcolor[rgb]{0.13,0.29,0.53}{\textbf{#1}}}
\newcommand{\DataTypeTok}[1]{\textcolor[rgb]{0.13,0.29,0.53}{#1}}
\newcommand{\DecValTok}[1]{\textcolor[rgb]{0.00,0.00,0.81}{#1}}
\newcommand{\DocumentationTok}[1]{\textcolor[rgb]{0.56,0.35,0.01}{\textbf{\textit{#1}}}}
\newcommand{\ErrorTok}[1]{\textcolor[rgb]{0.64,0.00,0.00}{\textbf{#1}}}
\newcommand{\ExtensionTok}[1]{#1}
\newcommand{\FloatTok}[1]{\textcolor[rgb]{0.00,0.00,0.81}{#1}}
\newcommand{\FunctionTok}[1]{\textcolor[rgb]{0.13,0.29,0.53}{\textbf{#1}}}
\newcommand{\ImportTok}[1]{#1}
\newcommand{\InformationTok}[1]{\textcolor[rgb]{0.56,0.35,0.01}{\textbf{\textit{#1}}}}
\newcommand{\KeywordTok}[1]{\textcolor[rgb]{0.13,0.29,0.53}{\textbf{#1}}}
\newcommand{\NormalTok}[1]{#1}
\newcommand{\OperatorTok}[1]{\textcolor[rgb]{0.81,0.36,0.00}{\textbf{#1}}}
\newcommand{\OtherTok}[1]{\textcolor[rgb]{0.56,0.35,0.01}{#1}}
\newcommand{\PreprocessorTok}[1]{\textcolor[rgb]{0.56,0.35,0.01}{\textit{#1}}}
\newcommand{\RegionMarkerTok}[1]{#1}
\newcommand{\SpecialCharTok}[1]{\textcolor[rgb]{0.81,0.36,0.00}{\textbf{#1}}}
\newcommand{\SpecialStringTok}[1]{\textcolor[rgb]{0.31,0.60,0.02}{#1}}
\newcommand{\StringTok}[1]{\textcolor[rgb]{0.31,0.60,0.02}{#1}}
\newcommand{\VariableTok}[1]{\textcolor[rgb]{0.00,0.00,0.00}{#1}}
\newcommand{\VerbatimStringTok}[1]{\textcolor[rgb]{0.31,0.60,0.02}{#1}}
\newcommand{\WarningTok}[1]{\textcolor[rgb]{0.56,0.35,0.01}{\textbf{\textit{#1}}}}
\usepackage{graphicx}
\makeatletter
\def\maxwidth{\ifdim\Gin@nat@width>\linewidth\linewidth\else\Gin@nat@width\fi}
\def\maxheight{\ifdim\Gin@nat@height>\textheight\textheight\else\Gin@nat@height\fi}
\makeatother
% Scale images if necessary, so that they will not overflow the page
% margins by default, and it is still possible to overwrite the defaults
% using explicit options in \includegraphics[width, height, ...]{}
\setkeys{Gin}{width=\maxwidth,height=\maxheight,keepaspectratio}
% Set default figure placement to htbp
\makeatletter
\def\fps@figure{htbp}
\makeatother
\setlength{\emergencystretch}{3em} % prevent overfull lines
\providecommand{\tightlist}{%
  \setlength{\itemsep}{0pt}\setlength{\parskip}{0pt}}
\setcounter{secnumdepth}{-\maxdimen} % remove section numbering
\ifLuaTeX
  \usepackage{selnolig}  % disable illegal ligatures
\fi
\usepackage{bookmark}
\IfFileExists{xurl.sty}{\usepackage{xurl}}{} % add URL line breaks if available
\urlstyle{same}
\hypersetup{
  pdftitle={NBES calculation in simulated data},
  hidelinks,
  pdfcreator={LaTeX via pandoc}}

\title{NBES calculation in simulated data}
\author{}
\date{\vspace{-2.5em}}

\begin{document}
\maketitle

Here we analyse the model runs created to explore the underlying
mechanisms in the NBES paper.

\subsubsection{Settings}\label{settings}

We created disturbed and undisturbed communities with.

n = 5 species;\\
repetitions = 20, 20 independent simulations for each scenario;\\
time points = 150;

disturbance Type: Press Disturbance\\
Temp Min: 15 (minimum temperature in press disturbance);\\
Temp Max:20 (maximum temperature in press disturbance);\\
Temp Control: 17.5 (temperature for control run).

We manipulated response diversity (RD) and species competition strength
(alpha). Specifically, we tested three levels of competitiveness in our
communities, that were drawn from a left-sided distribution, so that all
interactions are competitive: sd = 0 - no interaction; sd = 0.25
intermediate interaction strength; sd = 0.5 high interaction strength;

For response diversity, we manipulated the distribution of temperature
optimum within the community. Specificially, we tested: no RD: Topt of
15 and thus a negative effect;\\
low RD: Topt ranging between 17-18 degree; intermed RD: Topt ranging
between 16-19 degree; high RD: Topt ranging between 15-20 degree;

We ran 20 independent simulations for each of the settings.

\begin{Shaded}
\begin{Highlighting}[]
\FunctionTok{library}\NormalTok{(here)}
\end{Highlighting}
\end{Shaded}

\begin{verbatim}
## here() starts at /Users/charlottekunze/Desktop/phD/Exp22/MicrocosmExp22
\end{verbatim}

\begin{Shaded}
\begin{Highlighting}[]
\FunctionTok{library}\NormalTok{(tidyverse)}
\end{Highlighting}
\end{Shaded}

\begin{verbatim}
## -- Attaching core tidyverse packages ------------------------ tidyverse 2.0.0 --
## v dplyr     1.1.4     v readr     2.1.5
## v forcats   1.0.0     v stringr   1.5.1
## v ggplot2   3.5.1     v tibble    3.2.1
## v lubridate 1.9.3     v tidyr     1.3.1
## v purrr     1.0.2
\end{verbatim}

\begin{verbatim}
## -- Conflicts ------------------------------------------ tidyverse_conflicts() --
## x dplyr::filter() masks stats::filter()
## x dplyr::lag()    masks stats::lag()
## i Use the conflicted package (<http://conflicted.r-lib.org/>) to force all conflicts to become errors
\end{verbatim}

\begin{Shaded}
\begin{Highlighting}[]
\FunctionTok{library}\NormalTok{(ggbeeswarm)}

\NormalTok{nbes\_data100 }\OtherTok{\textless{}{-}} \FunctionTok{readRDS}\NormalTok{(}\StringTok{\textquotesingle{}\textasciitilde{}/Desktop/phD/Exp22/NBES{-}main/NBES/output/nbesSummary\_100.RData\textquotesingle{}}\NormalTok{)}\SpecialCharTok{\%\textgreater{}\%}
  \FunctionTok{mutate}\NormalTok{(}\AttributeTok{RD =} \FunctionTok{paste}\NormalTok{(}\StringTok{\textquotesingle{}high\textquotesingle{}}\NormalTok{))}


\CommentTok{\#add diversity level description}
\NormalTok{nbes\_plot }\OtherTok{\textless{}{-}}\NormalTok{ nbes\_data100 }\SpecialCharTok{\%\textgreater{}\%}
  \FunctionTok{mutate}\NormalTok{(}\AttributeTok{RD =}\NormalTok{ tOptUpper)}

\NormalTok{nbes\_plot}\SpecialCharTok{$}\NormalTok{RD[nbes\_plot}\SpecialCharTok{$}\NormalTok{RD}\SpecialCharTok{==}\DecValTok{15}\NormalTok{] }\OtherTok{\textless{}{-}} \StringTok{\textquotesingle{}No RD (Topt = 15)\textquotesingle{}}
\NormalTok{nbes\_plot}\SpecialCharTok{$}\NormalTok{RD[nbes\_plot}\SpecialCharTok{$}\NormalTok{RD}\SpecialCharTok{==}\DecValTok{18}\NormalTok{] }\OtherTok{\textless{}{-}} \StringTok{\textquotesingle{}Low RD (17\textless{}Topt\textless{}18)\textquotesingle{}}
\NormalTok{nbes\_plot}\SpecialCharTok{$}\NormalTok{RD[nbes\_plot}\SpecialCharTok{$}\NormalTok{RD}\SpecialCharTok{==}\DecValTok{19}\NormalTok{] }\OtherTok{\textless{}{-}} \StringTok{\textquotesingle{}Intermed RD (16\textless{}Topt\textless{}19)\textquotesingle{}}
\NormalTok{nbes\_plot}\SpecialCharTok{$}\NormalTok{RD[nbes\_plot}\SpecialCharTok{$}\NormalTok{RD}\SpecialCharTok{==}\DecValTok{20}\NormalTok{] }\OtherTok{\textless{}{-}} \StringTok{\textquotesingle{}High RD (15\textless{}Topt\textless{}20)\textquotesingle{}}

\DocumentationTok{\#\#\# create plots }\AlertTok{\#\#\#}

\CommentTok{\# NBES {-} Richness \#}
\NormalTok{nbes\_plot }\SpecialCharTok{\%\textgreater{}\%}
  \FunctionTok{group\_by}\NormalTok{(nSpecies, RD, compNormSd)}\SpecialCharTok{\%\textgreater{}\%}
  \FunctionTok{mutate}\NormalTok{(}\AttributeTok{mean.NBES =} \FunctionTok{mean}\NormalTok{(NBES),}
         \AttributeTok{sd.NBES =} \FunctionTok{sd}\NormalTok{(NBES))}\SpecialCharTok{\%\textgreater{}\%}
  \FunctionTok{ggplot}\NormalTok{(., }\FunctionTok{aes}\NormalTok{(}\AttributeTok{x=}\NormalTok{ nSpecies, }\AttributeTok{y =}\NormalTok{ NBES))}\SpecialCharTok{+}
    \FunctionTok{geom\_hline}\NormalTok{(}\AttributeTok{yintercept =} \DecValTok{0}\NormalTok{)}\SpecialCharTok{+}
    \FunctionTok{geom\_quasirandom}\NormalTok{(}\AttributeTok{size =} \FloatTok{0.7}\NormalTok{, }\AttributeTok{alpha =} \FloatTok{0.3}\NormalTok{)}\SpecialCharTok{+}
    \FunctionTok{geom\_errorbar}\NormalTok{(}\FunctionTok{aes}\NormalTok{(}\AttributeTok{ymin =}\NormalTok{ mean.NBES}\SpecialCharTok{{-}}\NormalTok{sd.NBES, }\AttributeTok{ymax =}\NormalTok{ mean.NBES}\SpecialCharTok{+}\NormalTok{sd.NBES), }\AttributeTok{width =}\NormalTok{ .}\DecValTok{1}\NormalTok{, }\AttributeTok{color =} \StringTok{\textquotesingle{}black\textquotesingle{}}\NormalTok{)}\SpecialCharTok{+}
    \FunctionTok{geom\_point}\NormalTok{(}\FunctionTok{aes}\NormalTok{(}\AttributeTok{y =}\NormalTok{ mean.NBES), }\AttributeTok{color =} \StringTok{\textquotesingle{}darkred\textquotesingle{}}\NormalTok{)}\SpecialCharTok{+}
   \FunctionTok{scale\_x\_continuous}\NormalTok{(}\AttributeTok{limits =} \FunctionTok{c}\NormalTok{(}\FloatTok{1.5}\NormalTok{,}\FloatTok{5.5}\NormalTok{),}\AttributeTok{breaks =} \FunctionTok{seq}\NormalTok{(}\DecValTok{2}\NormalTok{,}\DecValTok{5}\NormalTok{,}\DecValTok{1}\NormalTok{))}\SpecialCharTok{+}
    \FunctionTok{facet\_grid}\NormalTok{(RD}\SpecialCharTok{\textasciitilde{}}\NormalTok{compNormSd)}\SpecialCharTok{+}
    \FunctionTok{theme\_bw}\NormalTok{()}\SpecialCharTok{+}
    \FunctionTok{theme}\NormalTok{(}\AttributeTok{legend.position =} \StringTok{\textquotesingle{}none\textquotesingle{}}\NormalTok{)}
\end{Highlighting}
\end{Shaded}

\includegraphics{test_files/figure-latex/results-1.pdf}

We can observe that the NBES depends on both, species competition and
response diversity.

\begin{itemize}
\tightlist
\item
  The higher the competition strength between species, the larger is the
  total effect on NBES.
\item
  for high response diversity, we observe a positive trend with
  increasing richness
\item
  when there is no RD between species, the NBES becomes negative.
\end{itemize}

\begin{Shaded}
\begin{Highlighting}[]
\CommentTok{\# Competitive communities {-} NBES \#}
\NormalTok{nbes\_plot }\SpecialCharTok{\%\textgreater{}\%}
  \FunctionTok{filter}\NormalTok{(compNormSd }\SpecialCharTok{==}\FloatTok{0.5}\NormalTok{)}\SpecialCharTok{\%\textgreater{}\%}
  \FunctionTok{ggplot}\NormalTok{(., }\FunctionTok{aes}\NormalTok{(}\AttributeTok{x=}\NormalTok{ meanAlphas, }\AttributeTok{y =}\NormalTok{ NBES))}\SpecialCharTok{+}
  \FunctionTok{geom\_hline}\NormalTok{(}\AttributeTok{yintercept =} \DecValTok{0}\NormalTok{)}\SpecialCharTok{+}
  \FunctionTok{labs}\NormalTok{(}\AttributeTok{x =} \StringTok{\textquotesingle{}Mean realised Community Alpha\textquotesingle{}}\NormalTok{)}\SpecialCharTok{+}
  \FunctionTok{facet\_grid}\NormalTok{(RD}\SpecialCharTok{\textasciitilde{}}\NormalTok{nSpecies)}\SpecialCharTok{+}
  \FunctionTok{geom\_point}\NormalTok{(}\AttributeTok{size =} \FloatTok{0.7}\NormalTok{, }\AttributeTok{alpha =} \FloatTok{0.4}\NormalTok{)}\SpecialCharTok{+}
  \FunctionTok{geom\_smooth}\NormalTok{(}\AttributeTok{method =} \StringTok{\textquotesingle{}lm\textquotesingle{}}\NormalTok{)}\SpecialCharTok{+}
  \FunctionTok{theme\_bw}\NormalTok{()}\SpecialCharTok{+}
  \FunctionTok{ggtitle}\NormalTok{(}\StringTok{\textquotesingle{}Highly competitive communities\textquotesingle{}}\NormalTok{)}\SpecialCharTok{+}
  \FunctionTok{theme}\NormalTok{(}\AttributeTok{legend.position =} \StringTok{\textquotesingle{}none\textquotesingle{}}\NormalTok{)}
\end{Highlighting}
\end{Shaded}

\begin{verbatim}
## `geom_smooth()` using formula = 'y ~ x'
\end{verbatim}

\includegraphics{test_files/figure-latex/results2-1.pdf}

For communities with strong competition (sd = 0.5), we observe that the
NBES becomes more positive with increasing interspecific interactions.
This trend is even more pronounced, if the community has high response
diversity. For communities with no response diversity, the NBES becomes
negative with increasing interaction strength. This makes a lot of
sense, since with more species, we dont add any dissimilar responses to
the community but only more competition partners.

\begin{Shaded}
\begin{Highlighting}[]
\NormalTok{\# No competition {-} NBES \#}
\NormalTok{nbes\_plot \%\textgreater{}\%}
\NormalTok{  filter(compNormSd == 0)\%\textgreater{}\%}
\NormalTok{  ggplot(., aes(x= nSpecies, y = NBES))+}
\NormalTok{  facet\_grid(\textasciitilde{}RD)+}
\NormalTok{  ggtitle(\textquotesingle{}Communities without interaction\textquotesingle{})+}
\NormalTok{  geom\_hline(yintercept = 0)+}
\NormalTok{  geom\_quasirandom(size = 0.7, alpha = 0.3)+}
\NormalTok{  theme\_bw()}
\end{Highlighting}
\end{Shaded}

For communities without interaction (sd = 0), we observe that the NBES
is zero if also the RD is low/ zero. Thus, our metric works.

\end{document}
